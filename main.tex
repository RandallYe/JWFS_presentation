%\documentclass[10pt,t]{beamer}
%\documentclass[t,serif]{beamer}
%\documentclass[10pt,t,handout,serif,professionalfont]{beamer}
%\documentclass[11pt,t,serif,professionalfont,xcolor={dvipsnames}]{beamer}
\documentclass[%
% handout,% comment out for overlays
    slidestop,%
    compress,%
    mathserif,%
    table,%
    usenames,%
    aspectratio=169,
    dvipsnames,%
%noamsthm
]{beamer}%
%\documentclass[t,serif]{beamer}
%\setbeameroption{show notes}
%\setbeameroption{show only notes}

%%%% Notes on the second screen
%\usepackage{pgfpages}
%\setbeameroption{show notes on second screen}

\usepackage{main}

%%%%%%%%%%%%%%%%%%%%%%%%%%%%%%%%%%%%%%%%%%%%%%%%%%%%%%%%%%%%%%%%%%%%%%%%%%
% From RoboStar beamer template
%%%%%%%%%%%%%%%%%%%%%%%%%%%%%%%%%%%%%%%%%%%%%%%%%%%%%%%%%%%%%%%%%%%%%%%%%%

\mode<presentation>{
    \usetheme[compress]{Dresden}
    \definecolor{beamer@header}{HTML}{666666}
    \definecolor{beamer@blendedblue}{HTML}{015580}
    \definecolor{beamer@line}{HTML}{b5c55c}
    \definecolor{beamer@zed}{HTML}{0172ac}
    \definecolor{beamer@orange}{HTML}{ff4000}
    \setbeamercolor*{palette primary}{fg=white,bg=beamer@header}
    \setbeamercolor*{palette secondary}{fg=white,bg=beamer@line}

    \setbeamercolor{frametitle}{fg=beamer@blendedblue,bg=gray!10!white}
    \setbeamercolor{structure}{fg=beamer@header}
    \setbeamercolor{titlelike}{parent=palette primary,bg=white,fg=beamer@blendedblue}
    \setbeamercolor{navigation symbols}{fg=beamer@line, bg=beamer@header}
    \expandafter\def\expandafter\insertshorttitle\expandafter{%
        \insertshorttitle\hfill%
        \hfill\hspace{8cm}\insertframenumber\,/\,\inserttotalframenumber}
 % Comment the line below if you want to use the navigation symbols.
        \beamertemplatenavigationsymbolsempty
    }

    \defbeamertemplate*{title page}{customized}[1][]
    {
        \centering
        \vspace{1em}
        {\usebeamerfont{title}\usebeamercolor[fg]{frametitle}\inserttitle}\par
        \vspace{1em}
        {\usebeamerfont{subtitle}\usebeamercolor[fg]{subtitle}\insertsubtitle}\par
        \bigskip
        {\usebeamerfont{author}\usebeamercolor{titlelike}\insertauthor}\par

        \bigskip
        \vspace{2em}
        \includegraphics[scale=0.18]{pics/RoboStar_FC.eps}\par
        \vspace{0.5em}
        {\small \href{https://robostar.cs.york.ac.uk/}{robostar.cs.york.ac.uk}}\par
        \vspace{1em}
        {\tiny \usebeamerfont{date}\insertdate}\par
        \usebeamercolor[fg]{titlegraphic}\inserttitlegraphic
    }

%\usepackage[color]{circus}

%%%%%%%%%%%%%%%%%%%%%%%%%%%%%%%%%%%%%%%%%%%%%%%%%%%%%%%%%%%%%%%%%%%%%%%%%%
%%%%%%%%%%%%%%%%%%%%%%%%%%%%%%%%%%%%%%%%%%%%%%%%%%%%%%%%%%%%%%%%%%%%%%%%%%

% reduce margin
    \setbeamersize{text margin left=0.4cm,text margin right=0.4cm} 

%%%%%%%%%%%%%%%%%%%%%%%%%%%%%%%%%%%%%%%%%%
% Body 
%%%%%%%%%%%%%%%%%%%%%%%%%%%%%%%%%%%%%%%%%%
%\title[Automated Reasoning of Probabilistic Programs]{
%	Automated Reasoning for Probabilistic Sequential Programs\\ with Theorem Proving
%}
%\author[K. Ye, S. Foster, J. Woodcock]{\underline{Kangfeng Ye}, Simon Foster, Jim Woodcock}
%\institute[UoY]{\normalsize RoboStar\footnotemark[1]}

%\date{November 04, 2021}

%\titlegraphic{\includegraphics[align=c,width=4.2cm]{pics/UOY-Logo}\qquad\includegraphics[align=c,width=5.6cm]{pics/EPSRC_logo}}


    \begin{document}

%%%%%%%%%%%%%%%%%%%%%%%%%%%%%%%%%%%%%%%%%%%%%%%%%%%%%%%%%%%%%%%%%%%%%%%%%%%%%%%
%%%%%%%%%%%%%%%%%%%%%%%%%%%%%%%%%%%%%%%%%%%%%%%%%%%%%%%%%%%%%%%%%%%%%%%%%%%%%%%

\title[Programming choices]{{\Large A tour through the programming choices: semantics and applications}}
\subtitle{\large On the occasion of Jim Woodcock's Festschrift}


\author[\underline{Pedro Ribeiro} et al.]
{\normalsize \underline{Pedro Ribeiro}, \underline{Kangfeng Ye}, Frank Zeyda, Alvaro Miyazawa}

%\institute[University of York]{\normalsize University of York, UK}

\date{\small Sept 4, 2024}

%%%%%%%%%%%%%%%%%%%%%%%%%%%%%%%%%%%%%%%%%%%%%%%%%%%%%%%%%%%%%%%%%%%%%%
%%%%%%%%%%%%%%%%%%%%%%%%%%%%%%%%%%%%%%%%%%%%%%%%%%%%%%%%%%%%%%%%%%%%%

\frame{%
    \titlepage
    {%\centering
 %\medskip
  % Here you can include the UoY logo, for example. EPS format will 
  % typically be converted to PDF automatically if using pdflatex.
  %
  %\includegraphics[scale=.04]{pics/UoY-Logo.eps} \qquad
  %\hfill
 % \includegraphics[align=c,width=5.6cm]{pics/EPSRC_logo}
\includegraphics[align=c,width=2.6cm]{pics/UOY-Logo}
\hfill\hspace{4cm} \includegraphics[align=c,width=3.6cm]{pics/EPSRC_logo}
}

  %
  % See https://www.york.ac.uk/staff/external-relations/brand/logo/
  % for source of UoY logo.
  % 
  % Include other logos as appropriate.
  %
  %\smallskip
  %\href{https://robostar.cs.york.ac.uk}{robostar.cs.york.ac.uk}

    }%

    \logo{\includegraphics[scale=0.10]{pics/RoboStar_FC.eps}\quad~~}

    \setbeamerfont{footnote}{size=\tiny}

    \begin{frame}[fragile]
        \frametitle{Outline}
        \setlength{\parskip}{-1ex} %reduce vertical space between entries
        \begin{minipage}{\textwidth} % use minipage to reduce vertical space between entries
            \tableofcontents[hideallsubsections]
        \end{minipage}
    \end{frame}

% TOC
%\AtBeginSubsection[]
%\AtBeginSection[]
%{
%  \begin{frame}<beamer>
%    \frametitle{Outline}
%%    \tableofcontents[currentsection,currentsubsection]
%    \tableofcontents[currentsection, hideallsubsections]
%  \end{frame}
%}

% insert a TOC in the beginning of each section
\AtBeginSection[]
{
    \begin{frame}<beamer>
        \frametitle{Outline}
        \setlength{\parskip}{-1ex}
        \begin{minipage}{\textwidth} % use minipage to reduce vertical space between entries
            \tableofcontents[
                currentsection,
                hideallsubsections,
        %currentsubsection, 
        %hideothersubsections, 
                sectionstyle=show/shaded, % current section show and other sections shaded
         subsectionstyle=show/shaded/hide, % current section show, other subsections in this section shaded, subsections in other sections hide
            ] 
        \end{minipage}
    \end{frame}
}

% insert a TOC in the beginning of each subsection
%    \AtBeginSubsection[]
%    {
%        \begin{frame}<beamer>
%            \frametitle{Outline}
%            \tableofcontents[
%        %currentsection,
%        %currentsubsection, 
%        %hideothersubsections, 
%                sectionstyle=show/shaded, % current section show and other sections shaded
%                subsectionstyle=show/shaded/hide,  % current section show, other subsections in this section shaded, subsections in other sections hide
%            ] 
%        \end{frame}
%    }


\begin{frame}{Jim Woodcock}
    \begin{itemize}
        \item This talk is dedicated with affection to Prof. Jim Woodcook on his retirement from University of York.
    \end{itemize}

    % We've all had the pleasure to study and work with Jim for many years.
    % This paper is inspired by Jim's efforts to unify different programming and modelling paradigms.
    
    \begin{block}{Frank}

    \end{block}
    \begin{block}{Alvaro}

    \end{block}
    \begin{block}{Pedro}

    \end{block}
    \begin{block}{Kangfeng (Randall)}

    \end{block}
\end{frame}


\begin{frame}{Motivation and overview}
    % more compact and high-level diagram of Fig. 1
    % Applications: FMI, RoboChart
    \begin{itemize}
        \item 
    \end{itemize}
\end{frame}

\begin{frame}{Definitions of various choices}
    \begin{itemize}
        \item 
    \end{itemize}
\end{frame}

\section{Nondeterministic choice}
% One slide for each choice and each for application
\begin{frame}{Nondeterministic choice}
    \begin{itemize}
        \item 
    \end{itemize}
\end{frame}

\begin{frame}{Application}
% maybe discuss the role of non-determinism in RoboChart and the need for the cover wfc and the consequence of removing it.
    \begin{itemize}
        \item 
    \end{itemize}
\end{frame}

\section{Angelic choice}
\begin{frame}{Angelic choice}
    \begin{itemize}
        \item 
    \end{itemize}
\end{frame}

\begin{frame}{FMU/RoboChart}
    \begin{itemize}
        \item 
    \end{itemize}
\end{frame}

\section{Preferential choice}
\begin{frame}{Preferential choice}
    \begin{itemize}
        \item 
    \end{itemize}
\end{frame}

\begin{frame}{Application/go back to talk about FMU}
    \begin{itemize}
        \item 
    \end{itemize}
\end{frame}

\section{Probabilistic choice}
\begin{frame}{Probabilistic choice}
    \begin{itemize}
        \item 
    \end{itemize}
\end{frame}

\begin{frame}{Application/Random walk}
    \begin{itemize}
        \item 
    \end{itemize}
\end{frame}

\begin{frame}{Conclusion}
    \begin{itemize}
        \item 
    \end{itemize}
\end{frame}

\begin{frame}[c]{ }
    \usebeamerfont{frametitle}\usebeamercolor[fg]{frametitle}
    \centering 
    \Huge
    \emph{Thank you!} \\
    \vspace{0.5em}
    \normalsize
    \color{black}{https://robostar.cs.york.ac.uk/}
\end{frame}

%%%% Bibliography %%%%
\newpage
%\bibliographystyle{IEEEtran}
\bibliographystyle{alpha}
\bibliography{main} 
\label{ch:bib} %label to refer to
\end{document}
